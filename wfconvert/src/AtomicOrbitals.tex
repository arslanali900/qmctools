\documentclass{article}
\usepackage{amsmath}
\title{Notes on projecting to atomic orbitals}
\author{Ken Esler}
\date{July 19, 2009}
\begin{document}
\maketitle
\newcommand{\vr}{\mathbf{r}}
\newcommand{\vI}{\mathbf{I}}
\newcommand{\vk}{\mathbf{k}}
\newcommand{\vG}{\mathbf{G}}
\section{Form for orbitals}
Inside a muffin tin, orbitals are represented as product of spherical
harmonics and 1D radial functions, primarily represented by splines.
For a muffin tin centered at $\vI$, 
\begin{equation}
\phi_n(\vr) = \sum_{\ell,m} Y_\ell^m(\hat{\vr -\vI})
u_{lm}\left(\left|\vr - \vI\right|\right) \label{eq:ulm}
\end{equation}
Let use consider the case that our original representation for
$\phi(\vr)$ is of the form
\begin{equation}
\phi_{n,\vk}(\vr) = \sum_\vG c_{\vG+\vk}^n e^{i(\vG + \vk)\cdot \vr}
\end{equation}
Recall that
\begin{equation}
e^{i\vk\cdot\vr} = 4\pi \sum_{\ell,m} i^\ell j_\ell(|\vr||\vk|)
Y_\ell^m(\hat{\vk}) \left[Y_\ell^m(\hat{\vr})\right]^*.
\end{equation}
Conjugating,
\begin{equation}
e^{-i\vk\cdot\vr} = 4\pi\sum_{\ell,m} (-i)^\ell j_\ell(|\vr||\vk|)
\left[Y_\ell^m(\hat{\vk})\right]^* Y_\ell^m(\hat{\vr}).
\end{equation}
Setting $\vk \rightarrow -k$,
\begin{equation}
e^{i\vk\cdot\vr} = 4\pi\sum_{\ell,m} i^\ell j_\ell(|\vr||\vk|)
\left[Y_\ell^m(\hat{\vk})\right]^* Y_\ell^m(\hat{\vr}).
\end{equation}

Then,
\begin{equation}
e^{i\vk\cdot(\vr-\vI)} = 4\pi\sum_{\ell,m} i^\ell j_\ell(|\vr-\vI||\vk|)
\left[Y_\ell^m(\hat{\vk})\right]^* Y_\ell^m(\hat{\vr-\vI}).
\end{equation}

\begin{equation}
e^{i\vk\cdot\vr} = 4\pi e^{i\vk\cdot\vI} \-\sum_{\ell,m} i^\ell j_\ell(|\vr-\vI||\vk|)
\left[Y_\ell^m(\hat{\vk})\right]^* Y_\ell^m(\hat{\vr-\vI}).
\end{equation}

Then
\begin{equation}
\phi_{n,\vk}(\vr) =  \sum_\vG 4\pi c_{\vG+\vk}^n
e^{i(\vG+\vk)\cdot\vI} \sum_{\ell,m}
  i^\ell j_\ell(|\vG +\vk||\vr-\vI|)
  \left[Y_\ell^m(\hat{\vG+\vk})\right]^*
Y_\ell^m(\hat{\vr - \vI})
\end{equation}
Comparing to (\ref{eq:ulm}),
\begin{equation}
u_{\ell m}^n(r) = 4\pi i^\ell \sum_G c_{\vG+\vk}^n e^{i(\vG+\vk)\cdot\vI}  j_\ell\left(|\vG + \vk|r|\right)
\left[Y_\ell^m(\hat{\vG + \vk})\right]^*.
\end{equation}
If we had adopted the opposite sign convention for Fourier transforms
(as is unfortunately the case in wfconv), we woul have
\begin{equation}
u_{\ell m}^n(r) = 4\pi (-i)^\ell \sum_G c_{\vG+\vk}^n e^{-i(\vG+\vk)\cdot\vI}  j_\ell\left(|\vG + \vk|r|\right)
\left[Y_\ell^m(\hat{\vG + \vk})\right]^*.
\end{equation}



\end{document}

