\documentclass{article}
\usepackage{amsmath}
\usepackage{array}
\author{Kenneth P .Esler Jr.}
\title{Generalized band unfolding for quantum Monte Carlo simulation
  of solids}
\begin{document}
\maketitle
In continuum quantum Monte Carlo simulations, it is necessary to
evaluate the electronic orbitals of a system at real-space positions
hundreds of millions of times.  It has been found that if
these orbitals are represented in a localized, B-spline basis, each
evaluation takes a small, constant time which is independent of system
size.

Unfortunately, the memory required for storing the B-spline grows with
the second power of the system size.  If we are studying perfect
crystals, however, this can be reduced to linear scaling if we {\em
  tile} the primitive cell.  In this approach, implemented in the
CASINO QMC simulation suite, a supercell is constructed by tiling the
primitive cell $N_1 \times N_2 \times N_3$ in the three lattice
directions.  The orbitals are then represented in real space only in
the primitive cell, and an $N_1 \times N_2 \times N_3$ k-point mesh.
To evaluate an orbital at any point in the supercell, it is only
necessary to wrap that point back into the primitive cell, evaluate
the spline, and then multiply the phase factor,
$e^{-i\mathbf{k}\cdot\mathbf{r}}$.  

Here, we show that this approach can be generalized to a tiling
constructed with a $3\times 3$ nonsingular matrix of integers, of which
the above approach is a special case.  This generalization brings with
it a number of advantages.  The primary reason for performing
supercell calculations in QMC is to reduce finite-size errors.  These
errors result from three sources:  1) the quantization of the crystal
momentum;  2) the unphysical periodicity of the exchange-correlation
hole of the electron; and 3) the kinetic-energy contribution from the
periodicty of the long-range jastrow correlation functions.  The first
source of error can be largely eliminated by twist averaging.  If the
simulation cell is large enough that XC hole does not ``leak'' out of
the simulation cell, the second source can be eliminated either
through use of the MPC interaction or the {\em a postiori} correction
of Chiesa et. al.  

The satisfaction of the leakage requirement is controlled by whether
the minimum distance, $L_\text{min}$ from one supercell image to the
next is greater than the width of the XC hole.  Therefore, given a
choice, it is best to use a cell which is as nearly cubic as possible,
since this choice maximizes $L_\text{min}$ for a given number of
atoms.  Most often, however, the primitive cell is not cubic.  In
these cases, if we wish to choose the optimal supercell to reduce
finite size effects, we cannnot utilize the simple primitive tiling
scheme.  In the generalized scheme we present, it is possible to
choose far better supercells (from the standpoint of finite-size
errors), while retaining the storage efficiency of the original tiling
scheme.

\section{The mathematics}
\newcommand{\vp}{\mathbf{a}^\text{p}}
\newcommand{\vs}{\mathbf{a}^\text{s}} 
\newcommand{\Smat}{\mathbf{S}}
Consider the set of primitive lattice vectors, $\{\vp_1, \vp_2,
\vp_3\}$.  We may write these vectors in a matrix, $\mathbf{L}_p$, whose
rows are the primitive lattice vectors.  Consider a non-singular
matrix of integers, $\Smat$.  A corresponding set of supercell lattice
vectors, $\{\vs_1, \vs_2, \vs_3\}$, can be constructed by the matrix
product 
\begin{equation}
\vs_i = S_{ij} \vp_j
\end{equation}
If the primitive cell contains $N_p$ atoms, the supercell will then
contain $N_s = |\det(\Smat)| N_p$ atoms.

\section{Example: FeO}
As an example, consider the primitive cell for antiferromagnetic FeO
(wustite) in the rocksalt structure.  The primitive vectors, given in
units of the lattice constant, are given by
\newcommand{\xv}{\hat{\mathbf{x}}} 
\newcommand{\yv}{\hat{\mathbf{y}}}
\newcommand{\zv}{\hat{\mathbf{z}}}
\begin{eqnarray}
\vs_1 & = & \frac{1}{2}\xv + \frac{1}{2}\yv +      \ \   \zv \\
\vs_2 & = & \frac{1}{2}\xv +      \ \   \yv + \frac{1}{2}\zv \\
\vs_3 & = &   \ \      \xv + \frac{1}{2}\yv + \frac{1}{2}\zv 
\end{eqnarray}
This primitive cell contains two iron atoms and two oxygen atoms. It
is a very elongated cell with accute angles, and thus has a short
minimum distance between adjacent images.

The smallest cubic cell consistent with the AFM ordering can be
constructed with the matrix
\begin{equation}
\Smat = \left[\begin{array}{rrr}
  -1 & -1 &  3 \\
  -1 &  3 & -1 \\
   3 & -1 & -1 
  \end{array}\right]
\end{equation}
This cell has $2\det(\Smat) = 32$ iron atoms and 32 oxygen atoms.  In
this example, we may perform the simulation in the 32-iron supercell,
while storing the orbitals only in the 2-iron primitive cell, for a
savings of a factor of 16.  On current multicore supercomputers, with
1-2GB RAM per core, this is literally the difference between be able
to perform the simulation or not.

\subsection{The k-point mesh}
In order to be able to use the generalized tiling scheme, we need to
have the appropriate number of bands to occuppy in the supercell.
This may be achieved by appropriately choosing the k-point mesh.  In
this section, we explain how these points are chosen.  

For simplicity, let us assume that the supercell calculation will be
performed at the $\Gamma$-point.  We may lift this restriction very
easily later.  The fact that supercell calculation is performed at
$\Gamma$ implies that the k-points used in the primitive-cell
calculation must be $\mathbf{G}$-vectors of the superlattice.  This
still leaves us with an infinite set of vectors.  We may reduce this
set to a finite number by considering that the orbitals must form an
linearly independent set.  Orbitals with k-vectors $\mathbf{k}^p_1$
and $\mathbf{k}^p_2$ will differ by at most a constant factor if
$\mathbf{k}^p_1 - \mathbf{k}^p_2 = \mathbf{G}^p$, where $\mathbf{G}^p$
is a reciprocal lattice vector of the primitive cell.  

Combining these two considerations gives us a prescription for
generating our k-point mesh.  The mesh may be taken to be the set of
k-point which are G-vectors of the superlattice, reside within the
first Brillouin zone (FBZ) of the primitive lattice, whose members do
not differ a G-vector of the primitive lattice.  Upon constructing
such a set, we find that the number of included k-points is equal to
$|\det(\Smat)|$, precisely the number we need.  This can by considering
the fact that the supercell has a volume $|\det(\Smat)|$ times that of
the primitive cell.  This implies that the volume of the supercell's
FBZ is $|\det(\Smat)|^{-1}$ times that of the primitive cell.  Hence,
$|\det(\Smat)|$ G-vectors of the supercell will fit in the FBZ of the
primitive cell.  Removing duplicate k-vectors, which differ from
another by a reciprocal lattice vector, avoids double-counting vectors
which lie on zone faces.

\subsection{Formulae}
\newcommand{\Amat}{\mathbf{A}} 
\newcommand{\Bmat}{\mathbf{B}} 
\newcommand{\vk}{\mathbf{k}}
\newcommand{\vt}{\mathbf{t}}

Let $\Amat$ be the matrix whose rows are the direct lattice vectors,
$\{\mathbf{a}_i\}$.  The, let the matrix $\Bmat$ be defined as
$2\pi(\Amat^{-1})^\dagger$.  Its rows are the primitive reciprocal
lattice vectors.  Let $\Amat_p$ and $\Amat_s$ represent the primitive
and superlattice matrices, respectively, and similarly for their
reciprocals.  Then we have
\begin{eqnarray}
\Amat_s & = & \Smat \Amat_p \\
\Bmat_s & = & 2\pi\left[(\Smat \Amat_p)^{-1}\right]^\dagger \\
        & = & 2\pi\left[\Amat_p^{-1} \Smat{-1}\right]^\dagger \\
        & = & 2\pi(\Smat^{-1})^\dagger (\Amat_p^{-1})^\dagger \\
        & = & (\Smat^{-1})^\dagger \Bmat_p
\end{eqnarray}  
Consider a k-vector, $\vk$.  It may be alternatively be written in
basis of reciprocal lattice vectors as $\vt$.  
\begin{eqnarray}
\vk & = & (\vt^\dagger \Bmat)^\dagger \\
    & = & \Bmat^\dagger \vt           \\
\vt & = & (\Bmat^\dagger)^{-1} \vk    \\
    & = & (\Bmat^{-1})^\dagger \vk    \\
    & = & \frac{\Amat \vk}{2\pi}
\end{eqnarray}
We may then express a twist vector of the primitive lattice, $\vt_p$ in terms
of the superlattice.
\begin{eqnarray}
\vt_s & = & \frac{\Amat_s \vk}{2\pi}                           \\
      & = & \frac{\Amat_s \Bmat_p^\dagger \vt_p}{2\pi}         \\
      & = & \frac{\Smat \Amat_p \Bmat_p^\dagger \vt_p}{2\pi}   \\
      & = & \frac{2\pi \Smat \Amat_p \Amat_p^{-1} \vt_p}{2\pi} \\
      & = & \Smat \vt_p
\end{eqnarray}
This gives the simple result that twist-vectors transform in precisely
the same way as direct lattice vectors.



\end{document}

