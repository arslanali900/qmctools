\documentclass{revtex4}
\usepackage{color}
\usepackage{colordvi}
\usepackage{amsmath}
\usepackage{fancybox}
\usepackage{amssymb}
\usepackage{epsfig}
\newcommand{\rv}{\mathbf{r}}
\newcommand{\kv}{\mathbf{k}}
\newcommand{\Rv}{\mathbf{R}}
\newcommand{\Lv}{\mathbf{L}}
\newcommand{\Rc}{\mathcal{R}}
\newcommand{\tV}{\widetilde{V}}
\newcommand{\tW}{\widetilde{W}}
\newcommand{\tc}{\widetilde{c}}
\newcommand{\tY}{\widetilde{Y}}
\newcommand{\Nk}{N_\text{knot}}
\newcommand{\wk}{w_\text{knot}}

\newcommand{\trho}{\widetilde{\rho}}

\begin{document}

\title{Notes on fitpn}

\author{Simone Chiesa}\email{chiesa@uiuc.edu}
\affiliation{
Department of Physics, University of Illinois Urbana-Champaign, Urbana-Champaign, Illinois 61801}

\maketitle

Given a lattice of vectors $\Lv$, its associated reciprocal
lattice of vectors $\kv$ and a function $\psi(\rv)$ periodic
on the lattice we define its fourier transform $\widetilde{\psi}(\kv)$ as
\begin{equation}
\widetilde{\psi}(\kv)=\frac{1}{\Omega}\int_\Omega d\rv \psi(\rv) e^{-i\kv\rv}
\end{equation}
where we indicated both the cell domain and the cell volume by $\Omega$. 
$\psi(\rv)$ can then be expressed as
\begin{equation}
\psi(\rv)=\sum_{\kv} \widetilde{\psi}(\kv)e^{i\kv\rv}
\end{equation}
The potential generated by charges sitting on the lattice positions
at a particular point $\rv$ inside the cell is given by
\begin{equation}
V(\rv)=\sum_{\Lv}v(|\rv+\Lv|)
\end{equation}
and its fourier transform can be explicitely written as a function of $V$ or $v$
\begin{equation}
\widetilde{V}(\kv)=\frac{1}{\Omega}\int_\Omega d\rv V(\rv) e^{-i\kv\rv}=
\frac{1}{\Omega}\int_{\mathbb{R}^3} d\rv v(\rv) e^{-i\kv\rv}
\end{equation}
where $\mathbb{R}^3$ denotes the whole 3-dimensional space.
We now want to find the best (``best'' to be defined later) approximate 
potential of the form
\begin{equation}
V_a(\rv)=\sum_{k\le k_c} \widetilde{Y}(k) e^{i\kv\rv} + W(r)
\end{equation}
where $W(r)$ has been chosen to go to $0$ smoothly when $r=r_c$, being
$r_c$ lower or equal to the Wigner-Seitz radius of the cell. Note also
the cutoff $k_c$ on the momentum summation.

The best form of $\widetilde{Y}(k)$ and $W(r)$ is given by minimizing
\begin{equation}
\chi^2=\frac{1}{\Omega}\int d\rv \left(V(\rv)-W(\rv)-
\sum_{k\le k_c}\widetilde{Y}(k)e^{i\kv\rv}\right)^2
\label{chi2r}
\end{equation}
or the reciprocal space equivalent
\begin{equation}
\chi^2=\sum_{k\le k_c}(\tV(k)-\tW(k)-\tY(k))^2+\sum_{k>k_c}(\tV(k)-\tW(k))^2
\label{chi2k}
\end{equation}
Eq.\ref{chi2k} follows from Eq.\ref{chi2r} and the unitarity
(norm conservation) of the fourier transform.

This last condition is minimized by
\begin{equation}
\tY(k)=\tV(k)-\tW(k)\qquad \min_{\tW(k)}\sum_{k>k_c}(\tV(k)-\tW(k))^2
\label{mincond}
\end{equation}
We now use a set of basis function $c_i(r)$ vanishing smoothly at $r_c$
to expand $W(r)$ i.e.
\begin{equation}
W(r)=\sum_i t_i c_i(r)\qquad\text{or}\qquad \tW(k)=\sum_i t_i \tc_i(k)
\end{equation}
Inserting the reciprocal space expansion of $\tW$ in the second condition of
Eq.\ref{mincond} and minimizing with respect to $t_i$ leads immediately
to the linear system $\mathbf{A}\mathbf{t}=\mathbf{b}$ where
\begin{center}
\vskip 3mm
\fbox{
\begin{Beqnarray}
A_{ij}=\sum_{k>k_c}\tc_i(k)\tc_j(k)\qquad b_j=\sum_{k>k_c} V(k) \tc_j(k)
\label{matrix_elements}
\end{Beqnarray}
}
\end{center}
\vskip 3mm

\subsection*{Basis functions}
The basis functions are splines. We define a uniform grid 
with $\Nk$ uniformly spaced knots at position $r_i=i\frac{r_c}{\Nk}$ 
where $i\in[0,\Nk-1]$. On each knot we center $m+1$ piecewise polynomials
$c_{i\alpha}(r)$ with $\alpha\in[0,m]$, defined as
\vskip 3mm
\begin{center}
\fbox{
\begin{Beqnarray}
c_{i\alpha}(r)=\begin{cases}
\Delta^\alpha \sum_{n=0}^\mathcal{N} S_{\alpha n}(\frac{r-r_i}{\Delta})^n & r_i<r\le r_{i+1} \\
\Delta^{-\alpha} \sum_{n=0}^\mathcal{N} S_{\alpha n}(\frac{r_i-r}{\Delta})^n & r_{i-1}<r\le r_i \\
0 & |r-r_i| > \Delta
\end{cases}
\label{basisdef}
\end{Beqnarray}
}
\end{center}
\vskip 3mm
These functions and their derivatives are, by construction, continuous and odd (even)
(with respect to $r-r_i\rightarrow r_i-r$) when $\alpha$ is odd (even).
We further ask them to satisfy
\begin{eqnarray}
\left.\frac{d^\beta}{dr^\beta} c_{i\alpha}(r)\right|_{r=r_i}=
\delta_{\alpha\beta} \quad \beta\in[0,m]\\
\left.\frac{d^{\beta}}{dr^{\beta}} c_{i\alpha}(r)\right|_{r=r_{i+1}}=0\quad \beta\in[0,m]
\label{constr}
\end{eqnarray}
(The parity of the functions guarantees that the second constraint is satisfied
at $r_{i-1}$ as well). These constraints have a simple interpretation: the basis functions
and their first $m$ derivatives are $0$ on the boundary of the subinterval where they
are defined; the only function to have a non zero $\beta$-th derivative in $r_i$ is $c_{i\beta}$.
These $2(m+1)$ constraints therefore impose $\mathcal{N}=2m+1$. 
Inserting the definitions of Eq.(\ref{basisdef}) in the constraints of Eq.(\ref{constr})
leads to the set of $2(m+1)$ linear equation that fixes the value of $S_{\alpha n}$: 
\begin{eqnarray}
\Delta^{\alpha-\beta} S_{\alpha\beta} \beta!=\delta_{\alpha\beta}
\label{Smatrix1}\\
\Delta^{\alpha-\beta}\sum_{n=\beta}^{2m+1} S_{\alpha n} \frac{n!}{(n-\beta)!}=0
\end{eqnarray}
One can further simplify inserting the first of these equations into the second and write
the linear system as
\begin{equation}
\sum_{n=m+1}^{2m+1} S_{\alpha n} \frac{n!}{(n-\beta)!}=\begin{cases}
-\frac{1}{(\alpha-\beta)!}& \alpha\ge \beta \\
0 & \alpha < \beta
\end{cases}
\label{Smatrix2}
\end{equation}

\subsection*{Fourier components of the basis functions in 3D}
\subsubsection*{$k\ne 0$, non coulomb case.}
We now need to evaluate the fourier transforn $\tc_{i\alpha}(k)$. Let us start
by writing the definition
\begin{equation}
\tc_{i\alpha}(k)=\frac{1}{\omega}\int_\Omega d\rv  e^{-i\kv\rv} c_{i\alpha}(r)
\end{equation}
Because $c_{i\alpha}$ is different from zero only inside the spherical crown
defined by $r_{i-1}<r<r_i$ one can conveniently compute the integral in spherical
coordinates as
\vskip 3mm
\begin{center}
\fbox{
\begin{Beqnarray}
\tc_{i\alpha}(k)=\Delta^\alpha\sum_{n=0}^\mathcal{N} S_{\alpha n} \left[
D_{in}^+(k) +\wk(-1)^{\alpha+n}D_{in}^-(k)\right]
\label{fourier_transform}
\end{Beqnarray}
}
\end{center}
\vskip 3mm
where we used the definition $\wk=1-\delta_{i0}$ and
\begin{equation}
D_{in}^\pm(k)=\pm\frac{4\pi}{k\Omega}\Im\left[\int_{r_i}^{r_i\pm\Delta}
dr\left(\frac{r-r_i}{\Delta}\right)^n r e^{ikr}\right]
\label{D+-}
\end{equation}
obtained by integrating the angular part of the fourier transform.
Using the identity
\begin{equation}
\left(\frac{r-r_i}{\Delta}\right)^n r=\Delta\left(\frac{r-r_i}{\Delta}\right)^{n+1}+\left(\frac{r-r_i}{\Delta}\right)^n r_i
\end{equation}
and the definition
\begin{equation}
E_{in}^\pm(k)=\int_{r_i}^{r_i\pm\Delta}
dr\left(\frac{r-r_i}{\Delta}\right)^n e^{ikr}
\end{equation}
we rewrite Eq.\ref{D+-} as
\begin{center}
\vskip 3mm
\fbox{
\begin{Beqnarray}
D_{in}^\pm(k)=\pm\frac{4\pi}{k\Omega}\Im\left[\Delta E_{i(n+1)}^\pm(k)+
r_i E_{in}^\pm(k)\right]
\label{noncoulD+-}
\end{Beqnarray}
}
\end{center}
\vskip 3mm

Finally, using integration by part, one can define $E^\pm_{in}$ recursively
\begin{center}
\vskip 3mm
\fbox{
\begin{Beqnarray}
E^\pm_{in}(k)=\frac{1}{ik}\left[(\pm)^ne^{ik(r_i\pm\Delta)}-\frac{n}{\Delta}
E^\pm_{i(n-1)}(k)\right]
\label{nthEpm}
\end{Beqnarray}
}
\end{center}
\vskip 3mm
\noindent
starting from the $n=0$ term
\vskip 3mm
\begin{center}
\fbox{
\begin{Beqnarray}
E^\pm_{i0}(k)=\frac{1}{ik}e^{ikr_i}\left(e^{\pm ik\Delta}-1\right)
\label{0thEpm}
\end{Beqnarray}
}
\end{center}
\vskip 3mm
\subsubsection*{$k\ne 0$, coulomb case.}
To efficiently treat the coulomb divergence at the origin it is convenient to use
a basis set $c_{i\alpha}^\text{coul}$ of the form 
\begin{equation}
c_{i\alpha}^\text{coul}=\frac{c_{i\alpha}}{r}
\end{equation}
An equation identical to Eq.\ref{D+-} holds but with the modified definition
\begin{equation}
D_{in}^\pm(k)=\pm\frac{4\pi}{k\Omega}\Im\left[\int_{r_i}^{r_i\pm\Delta}
dr\left(\frac{r-r_i}{\Delta}\right)^n e^{ikr}\right]
\end{equation}
which can be simply expressed using $E^\pm_{in}(k)$ as
\vskip 3mm
\begin{center}
\fbox{
\begin{Beqnarray}
D_{in}^\pm(k)=\pm\frac{4\pi}{k\Omega}\Im\left[E_{in}^\pm(k)\right]
\label{coulD+-}
\end{Beqnarray}
}
\end{center}
\vskip 3mm
\subsubsection*{$k=0$ coulomb and non coulomb case.}
The definitions of $D_{in}(k)$ given so far are clearly incompatible 
with the choice $k=0$ (they involve division by $k$). For the non-coulomb
case the starting definition is
\begin{equation}
D^\pm_{in}(0)=\pm\frac{4\pi}{\Omega}\int_{r_i}^{r_i\pm\Delta}r^2
\left(\frac{r-r_i}{\Delta}\right)^ndr
\end{equation}
Using the definition $I_n^\pm=(\pm)^{n+1}\Delta/(n+1)$ we can express this
as
\begin{center}
\vskip 3mm
\fbox{
\begin{Beqnarray}
D^\pm_{in}(0)=\pm\frac{4\pi}{\Omega}\left[\Delta^2 I_{n+2}^\pm
+2r_i\Delta I_{n+1}^\pm+2r_i^2I_n^\pm\right]
\label{noncoul_k=0D+-}
\end{Beqnarray}
}
\end{center}
\vskip 3mm
For the coulomb case one get

\vskip 3mm
\begin{center}
\fbox{
\begin{Beqnarray}
D^\pm_{in}(0)=\pm\frac{4\pi}{\Omega}\left(
\Delta I^\pm_{n+1} + r_i I^\pm_n\right)
\label{coul_k=0D+-}
\end{Beqnarray}
}
\end{center}
\vskip 3mm
\subsection*{Fourier components of the basis functions in 2D}
Eq.\ref{fourier_transform} still holds provided we define  
\begin{equation}
D^\pm_{in}(k)=\pm\frac{2\pi}{\Omega \Delta^n} \sum_{j=0}^n \binom{n}{j}
(-r_i)^{n-j}\int_{r_i}^{r_i\pm \Delta}\negthickspace \negthickspace 
\negthickspace \negthickspace \negthickspace \negthickspace \negthickspace 
dr r^{j+1-C} J_0(kr)
\label{2DD+-}
\end{equation}
where $C=1(=0)$ for the coulomb(non coulomb) case.
Eq.\ref{2DD+-} is obtained using the integral definition of the 
zero order Bessel function of the first kind 
\begin{equation}
J_0(z)=\frac{1}{\pi}\int_0^\pi e^{iz\cos\theta}d\theta
\end{equation}
and the binomial expansion for $(r-r_i)^n$.
The integrals can be computed recursively using the following identitities
\begin{center}
\fbox{
\begin{minipage}{0.7\textwidth}
\begin{align}
&\int dz J_0(z)=\frac{z}{2}\left[\pi J_1(z)H_0(z)+J_0(z)(2-\pi H_1(z))\right]
\label{0thmoment}\\
&\int dz z J_0(z)= z J_1(z)
\label{1stmoment}\\
&\int dz z^n J_0(z)= z^nJ_1(z)+(n-1)x^{n-1}J_0(z)
-(n-1)^2\int dz z^{n-2} J_0(z)
\label{nthmoment}
\end{align}
\end{minipage}
}
\end{center}
Eq.\ref{nthmoment} is obtained using Eq.\ref{1stmoment}, integration by part and 
the identity $\int J_1(z) dz =-J_0(z)$. In Eq.\ref{0thmoment} $H_0$ and $H_1$ are Struve functions.

\subsection{Construction of the matrix elements}
Using the above equations one can construct the matrix elements in Eq.\ref{matrix_elements}
and proceed solving for the $t_i$. It is sometimes desirable to put some constraints
on the value of $t_i$. For example, when the coulomb potential is concerned one may 
want to set $t_{0}=1$. If the first $g$ variable are constrained by $t_{m}=\gamma_m$ 
with $m=[1,g]$ one can simply redefine Eq.\ref{matrix_elements} as
\begin{equation}
\begin{split}
A_{ij}=&\sum_{k>k_c} \tc_i(k)\tc_j(k)  \quad i,j\notin[1,g] \\
b_j=&\sum_{k>k_c} \left(\tV(k)-\sum_{m=1}^g \gamma_m \tc_m(k)\right)\tc_j(k)\quad j\notin[1,g]
\end{split}
\label{modified_matrix_elements}
\end{equation}

\subsection*{The routines}
\subsubsection*{fitpnnew}
This routine constructs the $t_i$ and $\tY(k)$. Previously a routine, 
let us call it {\em shells}, generating a grid of $\kv$ points has to
be called. {\em shells} stores $\kv$ vectors
in order of increasing magnitude and defines a shell as the 
set of vectors having the same magnitude $k$ (in practice their difference 
in magnitude must be below a given threshold). The total number of
shells $N_\text{shell}$ has to be large enough to represents $V(\rv)$
accurately using $\tV(k)$. The number of vector
in a given shell is called $w(k)$. The following variables are passed as
input: $\tV(k),k,w(k),N_\text{shell},m,r_c,N_\text{knot},\Omega$ and are called
\verb!v(0:nk),rk(0:nk),wt(0:nk),nk,m,rad,nknots!. Note that the vectors all
start from $0$ which corresponds to $k=0$. The number of shells such that
$k\le k_c$ is also passed as input and called \verb!nf!. Additional input variables 
are  \verb!coul! a logical variable specifying if the potential is coulombic; 
\verb!vmad! the exact value of the madelung constant;
\verb!t0,t1! logical variables specifying if a constraint has to be put
on element $t_0$ or $t_1$ and \verb!vt0,vt1! the value at which $t_0$ and $t_1$
have to be set if corresponding constraints are active. 
The routine works in this way:
\begin{itemize}
\item it calls {\em basis} and gets the coefficients $S_{\alpha n}$ (the $n$-th
      coefficient of the $\alpha$-th polynomials) for the desired value of $m$.
\item for every $k$ point, knot $i$ and polynomial $\alpha$ compute $\tc_{i\alpha}(k)$
      using Eq.\ref{fourier_transform}. $D^\pm_{in}(k)$ is provided by {\em splint3D}
      or {\em splint2D}. The routine uses \verb!ialpha!$=i(m+1)+\alpha$ (the range of 
      variability of $\alpha$ and $i$ is specified aboce Eq.\ref{basisdef}).
\item Matrix elements are constructed according to Eq.\ref{matrix_elements}
\item Matrix elements are modified according to Eq.\ref{modified_matrix_elements} 
      if constraints are active
\item $t_i$ are computed solving the linear system. $\tY(k)$ are computed.
\item A comparison with the exact madelung constant is performed.
\end{itemize}

\subsubsection*{splint3D}
Called by {\em fitpnnew}. This routine compute $D^\pm_{in}(k)$ for given 
$k$ and $i$ and for all
$n$ (going from $0$ to $\mathcal{N}=2m+1$). $D^\pm_{in}(k)$ are called 
\verb!ddplus(0:maxn)! and \verb!ddminus(0:maxn)! and are given as output
by the routine. In input one is required to specify $\mathcal{N},r_i,\Delta,k,\Omega$,
respectively named \verb!maxn,r,delta,k,vol!. A logical input flag called 
\verb!coul! specify if the potential is coulombic or not. The routine works
in this way:
\begin{itemize}
\item it checks if $k$ is equal to 0
\item if $k\ne 0$ then
  \begin{itemize}
  \item it computes $E^\pm_{in}(k)$ for the specified $i$ and $k$ using Eqs.\ref{nthEpm} and
      \ref{0thEpm}. $E^\pm_{in}(k)$ are called \verb!ee(0:maxn,!$\pm$\verb!1)! 
      (\verb!ee(:,0)! are never used).
  \item Depending on the value of \verb!coul! either Eq.\ref{noncoulD+-} or Eq.\ref{coulD+-} 
      is used to construct $D^\pm_{in}(k)$. The prefactor $\frac{4\pi}{k \Omega}$ 
      is precomputed and called \verb!dnorm!.
  \end{itemize}
\item if $k=0$ the code uses either Eq.\ref{noncoul_k=0D+-} or Eq.\ref{coul_k=0D+-}.
\end{itemize}

\subsubsection*{splint2D}
Called by {\em fitpnnew}. This routine compute $D^\pm_{in}(k)$ in the 2D case.
The \verb!i\o! format is identical to {\em splint3D}. Equations from \ref{0thmoment}
to \ref{nthmoment} are used to generate the required integrals.


\subsubsection*{basis}
Called by {\em fitpnnew}. It computes the coefficients $S_{\alpha n}$ (the $n$-th 
coefficient of the $\alpha$-th polynomials) using Eqs.\ref{Smatrix1} and \ref{Smatrix2}.
These coefficients are stored in \verb!s(0:m,0:2m+1)!. $m$ (called \verb!m!) 
is required in input.

\subsubsection*{computespl}
This compute $W(r)$ at any $r$. $r$ is named \verb!rpos! internally. It requires
$m,2m+1,N_\text{knot},S_{\alpha n},r_i,t_i,\Delta$. These are internally called
\verb!m,maxn,nknots,s(0:m,0:maxn),r(0:nknots),t(0:nknots(m+1)-1),delta!. 
\verb!coul! is also needed: it is a logical variable 
to specify if $c_{i\alpha}^\text{coul}(r)$ have to be used instead of $c_{i\alpha}(r)$.
The value of $W(r)$ is stored in \verb!w!.

\end{document}
